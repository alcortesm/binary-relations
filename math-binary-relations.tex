\documentclass[11pt]{article}

\usepackage[english]{babel}
\usepackage[utf8]{inputenc}
\usepackage[T1]{fontenc}

\usepackage{amsmath}
\usepackage{amssymb}
\usepackage{amsthm}
\theoremstyle{plain}
\newtheorem{defn}{Definition}
\theoremstyle{definition}
\newtheorem*{example}{Example}

\usepackage{multicol}
\usepackage{enumitem}

\usepackage{color}
\usepackage{graphicx}
\usepackage{epstopdf}

\title{Some important binary relations}
\author{Alberto Cortés \\ {\texttt <alcortesm@gmail.com>}}
\date{\today}

\begin{document}

\maketitle

\section{Binay Relations}

In mathematics, a binary relation on a set $S$ is a collection of ordered pairs of elements of $A$.
In other words, it is a subset of the Cartesian product $A^2 = A \times A$.
More generally, a binary relation between two sets $S_1$ and $S_2$ is $\subseteq S_1 \times S_2$.

\begin{example}
  Let us say Alice is older than Bob, who is older than Carol,
  denoted by $a$, $b$, and $c$ from now on.
  If relation $R$ on set $S = \{a, b, c\}$ means ``is older than'', then $aRb, bRc$.
\end{example}

\begin{example}
  Let us say relation $R = \text{``is smaller than''}$ on set $\mathbb{N}$,
  then $\forall x \in \mathbb{N}, xR(x+1) \text{ and } \neg xRx$.
\end{example}

As binary relations are basically pairs, they can drawn as directed graphs.
For instance, let $S = \{a, b\}$ and relation $R \subseteq S \times S$. Then $aRb$ can be drawn as:
\begin{center}
  \input{simple-graph.pdf_t}
\end{center}


\section{In Computer Science}

TODO: explain computer representations: pairs and matrix.


\section{Properties}

\subsection{Reflexivity}

\begin{defn}
  Relation $R$ is \textbf{reflexive} $\iff \forall a \in S, aRa$.
  For example ``is equal to'', ``is as tall as''.
\end{defn}

\begin{defn}
  Relation $R$ is \textbf{irreflexive} $\iff \forall a \in S, \neg aRa$.
  For example ``is greater than'', ``is to the left of''.
\end{defn}

\begin{defn}
  Relation $R$ is \textbf{non-reflexive} $\iff R \text{ is neither reflexive nor irreflexive}$.
  For example ``is grandfather of'', as you can be your own granfather,
  see Ray Stevens song ``I am my own granpa'')
\end{defn}

\subsection{Transitivity}

\begin{defn}
  Relation $R$ is \textbf{transitive} $\iff \forall a \in S, aRa$.
\end{defn}

\begin{defn}
  Relation $R$ is \textbf{intransitive} $\iff \forall a \in S, aRa$.
\end{defn}

\begin{defn}
  Relation $R$ is \textbf{nontransitive} $\iff \forall a \in S, aRa$.
\end{defn}


\subsection{Symmetricity}

\begin{defn}
  Relation $R$ is \textbf{symmetric} $\iff \forall a \in S, aRa$.
\end{defn}

\begin{defn}
  Relation $R$ is \textbf{asymmetric} $\iff \forall a \in S, aRa$.
\end{defn}

\begin{defn}
  Relation $R$ is \textbf{antisymmetric} $\iff \forall a \in S, aRa$.
\end{defn}

\begin{defn}
  Relation $R$ is \textbf{nonsymmetric} $\iff \forall a \in S, aRa$.
\end{defn}

\subsection{Equivalence}

\begin{defn}
  Relation $R$ is \textbf{equivalent} $\iff \forall a \in S, aRa$.
\end{defn}


\section{Transitive Clousures}

TODO: explain transitive clousures.

\end{document}
